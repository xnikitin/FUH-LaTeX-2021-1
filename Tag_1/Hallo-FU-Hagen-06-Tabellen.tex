\documentclass[ngerman,12pt]{scrartcl}

\usepackage[utf8]{inputenc}
\usepackage[T1]{fontenc}
\usepackage{babel}

\usepackage{palatino}
\usepackage{microtype}
\usepackage{paralist}
\usepackage{blindtext,booktabs}

\usepackage{graphicx} % nicht graphics
% JPG, PNG, PDF

\usepackage{hyperref}
\hypersetup{
    bookmarks=true,                     % show bookmarks bar
    unicode=false,                      % non - Latin characters in Acrobat’s bookmarks
    pdftoolbar=true,                        % show Acrobat’s toolbar
    pdfmenubar=true,                        % show Acrobat’s menu
    pdffitwindow=false,                 % window fit to page when opened
    pdfstartview={FitH},                    % fits the width of the page to the window
    pdftitle={My title},                        % title
    pdfauthor={Author},                 % author
    pdfsubject={Subject},                   % subject of the document
    pdfcreator={Creator},                   % creator of the document
    pdfproducer={Producer},             % producer of the document
    pdfkeywords={keyword1, key2, key3},   % list of keywords
    pdfnewwindow=true,                  % links in new window
    colorlinks=true,                        % false: boxed links; true: colored links
    linkcolor=blue,                          % color of internal links
    filecolor=cyan,                     % color of file links
    citecolor=green,                     % color of file links
    urlcolor=magenta                        % color of external links
}



\title{Mein erstes \LaTeX-Dokument}
\author{Uwe Ziegenhagen}
\date{Köln, den 11.06.2021}


\begin{document}
\maketitle

\tableofcontents

\listoffigures

\listoftables

\section{Einleitung}

Hallo Fernuni Hagen! Siehe Tabelle \ref{tab:meineTabelle} auf Seite \pageref{tab:meineTabelle}.

Ein Editor zur Bearbeitung der TeX-Dateien: TeX Live bringt für Mac und Windows TeXworks mit, einen guten Editor, den ich gern benutze. 
TeX Studio und Visual Studio Code (mit der LaTeX Workshop Erweiterung von James Yu) kann ich ebenfalls sehr empfehlen.

\includegraphics[width=5cm]{Bilder/miau}


\begin{figure}[t] % h = here, t = top, b = bottom
\centering
\includegraphics[width=0.75\textwidth]{Bilder/melli} % Textbreite
\caption{Ich bin Melli}\label{fig:melli2}
\end{figure}


\blindtext 

\begin{tabular}{lrcp{5cm}}
Hallo & ich bin eine & Tabellenzelle & Ein Editor zur Be\-ar\-beitung der TeX-Dateien: TeX Live bringt für Mac und Windows TeXworks mit, einen guten Editor, den ich gern benutze.  \\
Ich & bin & toll & Ein Editor zur Bearbeitung der TeX-Dateien: TeX Live bringt für Mac und Windows TeXworks mit, einen guten Editor, den ich gern benutze.  \\
\end{tabular}

\begin{tabular}{|l|r|c|p{6cm}|}
Hallo & ich bin eine & Tabellenzelle & Ein Editor zur Be\-ar\-beitung der TeX-Dateien: TeX Live bringt für Mac und Windows TeXworks mit, einen guten Editor, den ich gern benutze.  \\
Ich & bin & toll & Ein Editor zur Bearbeitung der TeX-Dateien: TeX Live bringt für Mac und Windows TeXworks mit, einen guten Editor, den ich gern benutze.  \\
\end{tabular}

\begin{tabular}{|l|r|c|p{6cm}|}
Hallo & ich bin eine & Tabellenzelle & Ein Editor zur Be\-ar\-beitung der TeX-Dateien: TeX Live bringt für Mac und Windows TeXworks mit, einen guten Editor, den ich gern benutze.  \\
Ich & bin & toll & Ein Editor zur Bearbeitung der TeX-Dateien: TeX Live bringt für Mac und Windows TeXworks mit, einen guten Editor, den ich gern benutze.  \\
\end{tabular}

\vspace*{2cm}

\begin{tabular}{|l|r|c|p{6cm}|} \hline
Spalte 1 & Spalte 2 & Spalte 3 & Spalte 4 \\ \hline
Hallo & ich bin eine & Tabellenzelle & Ein Editor zur Be\-ar\-beitung der TeX-Dateien: TeX Live bringt für Mac und Windows TeXworks mit, einen guten Editor, den ich gern benutze.  \\ \hline
Ich & bin & toll & Ein Editor zur Bearbeitung der TeX-Dateien: TeX Live bringt für Mac und Windows TeXworks mit, einen guten Editor, den ich gern benutze.  \\ \hline
\end{tabular}

Vertikale Linien in Tabellen sollten tunlichst vermieden werden, Tabellen in Satzqualität erhält man mit dem booktabs Paket.

\section*{Tabellen mit booktabs}
% * bedeutet, nicht ins Inhaltsverzeichnis

Vertikale Linien raus!

\begin{tabular}{lrcp{6cm}} \hline
Spalte 1 & Spalte 2 & Spalte 3 & Spalte 4 \\ \hline
Hallo & ich bin eine & Tabellenzelle & Ein Editor zur Be\-ar\-beitung der TeX-Dateien: TeX Live bringt für Mac und Windows TeXworks mit, einen guten Editor, den ich gern benutze.  \\ \hline
Ich & bin & toll & Ein Editor zur Bearbeitung der TeX-Dateien: TeX Live bringt für Mac und Windows TeXworks mit, einen guten Editor, den ich gern benutze.  \\ \hline
\end{tabular}

\vspace*{1cm}

Statt hlines toprule und bottomrule Befehle nutzen

\vspace*{1cm}

\begin{tabular}{lrcp{6cm}} \toprule[1.5pt]
Spalte 1 & Spalte 2 & Spalte 3 & Spalte 4 \\ \hline
Hallo & ich bin eine & Tabellenzelle & Ein Editor zur Be\-ar\-beitung der TeX-Dateien: TeX Live bringt für Mac und Windows TeXworks mit, einen guten Editor, den ich gern benutze.  \\ \hline
Ich & bin & toll & Ein Editor zur Bearbeitung der TeX-Dateien: TeX Live bringt für Mac und Windows TeXworks mit, einen guten Editor, den ich gern benutze.  \\ \bottomrule[2.5pt]
\end{tabular}

\vspace*{1cm}

Statt hlines in der Tabelle midrule

\pagebreak

\begin{tabular}{lrcp{6cm}} \toprule[1.5pt]
Spalte 1 & Spalte 2 & Spalte 3 & Spalte 4 \\ \midrule
Hallo & ich bin eine & Tabellenzelle & Ein Editor zur Be\-ar\-beitung der TeX-Dateien: TeX Live bringt für Mac und Windows TeXworks mit, einen guten Editor, den ich gern benutze.  \\ \midrule
Ich & bin & toll & Ein Editor zur Bearbeitung der TeX-Dateien: TeX Live bringt für Mac und Windows TeXworks mit, einen guten Editor, den ich gern benutze.  \\ \bottomrule[2.5pt]
\end{tabular}

\vspace*{1cm}

Statt midrule cmidrule für linke bzw. rechte Einrückung

\vspace*{1cm}

\begin{tabular}{lrcp{6cm}} \toprule[1.5pt]
Spalte 1 & Spalte 2 & Spalte 3 & Spalte 4 \\ \cmidrule[1pt](rl){1-4}
Hallo & ich bin eine & Tabellenzelle & Ein Editor zur Be\-ar\-beitung der TeX-Dateien: TeX Live bringt für Mac und Windows TeXworks mit, einen guten Editor, den ich gern benutze.  \\ \cmidrule[1pt](rl){1-4}
Ich & bin & toll & Ein Editor zur Bearbeitung der TeX-Dateien: TeX Live bringt für Mac und Windows TeXworks mit, einen guten Editor, den ich gern benutze.  \\ \bottomrule[2.5pt]
\end{tabular}

\vspace*{1cm}

\begin{table}
\centering
\caption{Ich bin eine Tabelle}\label{tab:meineTabelle}
\begin{tabular}{lrcp{6cm}} \toprule[1.5pt]
\textbf{Spalte 1} & \textbf{Spalte 2} & \textbf{Spalte 3} & \textbf{Spalte 4} \\ \cmidrule[1pt](rl){1-4}
Hallo & ich bin eine & Tabellenzelle & Ein Editor zur Be\-ar\-beitung der TeX-Dateien: TeX Live bringt für Mac und Windows TeXworks mit, einen guten Editor, den ich gern benutze.  \\ \midrule[0.5pt]
Ich & bin & toll & Ein Editor zur Bearbeitung der TeX-Dateien: TeX Live bringt für Mac und Windows TeXworks mit, einen guten Editor, den ich gern benutze.  \\ \bottomrule[2.5pt]
\end{tabular}
\end{table}

\end{document}




