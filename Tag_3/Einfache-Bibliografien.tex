\documentclass[12pt,ngerman,parskip=half]{scrartcl}

\usepackage[utf8]{inputenc} % Eingabekodierung
\usepackage[T1]{fontenc} % Schriftkodierung 
\usepackage{booktabs} % schöne Tabellen
\usepackage{babel} % Silbentrennung und ``Eindeutschung''
\usepackage{graphicx} % Grafiken einbinden
\usepackage{csquotes} % Zitate
\usepackage{paralist} % Kompakte Aufzählungen
\usepackage{xcolor} % Farben
\usepackage{amsmath} % Erweiterung des Mathematik-Modus
\usepackage{blindtext} % Dummytext
\usepackage{microtype} %Mikrotypografie

% ohne Paket ==> Standard-Schrift = Computer Modern
\usepackage{lmodern} % Moderne Version der Computer Modern



\title{Schriftarten in pdf\LaTeX}
\author{Uwe Ziegenhagen}

\begin{document}
\maketitle


\blindtext

\begin{equation}
x_{1,2} = -\frac{p}{2} \pm \sqrt{  \left( \frac{p}{2} \right) ^2 - q  }
\end{equation}


\blindtext

\begin{equation}
\left\{
\begin{array}{l}
p = - \frac{5}{3} P^3 \left\{A \frac{\delta^2 \left(\frac{1}{\rho} \right)}{\delta \xi^2} + B  \frac{\delta^2 \left(\frac{1}{\rho} \right)}{\delta \eta^2} + C \frac{\delta^2 \left(\frac{1}{\rho} \right)}{\delta \zeta^2} \right\} + \text{konst.}, \\
u = A \xi - \frac{5}{3} P^2 A \frac{\xi}{\rho^3} - \frac{\delta D}{\delta \xi}, \\
v = B \eta - \frac{5}{3} P^3 B \frac{\eta}{\rho^3} - \frac{\delta D}{\delta \eta}, \\
w = C \zeta - \frac{5}{3} P^3 C \frac{\zeta}{\rho^3} - \frac{\delta D}{\delta \zeta},
\end{array}
\right.
\end{equation}


\blindtext

Donald Knuth hat in \cite{knuth} gezeigt, das $P=NP$. 
%Karl Marx hat in \cite{marx} auch einiges gezeigt.

\blindtext

\begin{thebibliography}{19}

\bibitem{knuth}
\textbf{Donald E. Knuth}, \textit{The \TeX book}, Addison Wesley, Massachusetts, 2nd edition, 1984

\bibitem{marx}
\textbf{Karl Max}, \textit{Das Kapital}, Vorwärts-Verlag Berlin, 1st edition,1850

\bibitem{marx3}
\textbf{Karl Max}, \textit{Das Kapital}, Vorwärts-Verlag Berlin, 1st edition,1850

\bibitem{marx4}
\textbf{Karl Max}, \textit{Das Kapital}, Vorwärts-Verlag Berlin, 1st edition,1850

\bibitem{marx5}
\textbf{Karl Max}, \textit{Das Kapital}, Vorwärts-Verlag Berlin, 1st edition,1850

\bibitem{marx6}
\textbf{Karl Max}, \textit{Das Kapital}, Vorwärts-Verlag Berlin, 1st edition,1850

\bibitem{marx7}
\textbf{Karl Max}, \textit{Das Kapital}, Vorwärts-Verlag Berlin, 1st edition,1850

\bibitem{marx8}
\textbf{Karl Max}, \textit{Das Kapital}, Vorwärts-Verlag Berlin, 1st edition,1850

\bibitem{marx9}
\textbf{Karl Max}, \textit{Das Kapital}, Vorwärts-Verlag Berlin, 1st edition,1850

\bibitem{marx10}
\textbf{Karl Max}, \textit{Das Kapital}, Vorwärts-Verlag Berlin, 1st edition,1850

\bibitem{marx11}
\textbf{Karl Max}, \textit{Das Kapital}, Vorwärts-Verlag Berlin, 1st edition,1850


\end{thebibliography}


\end{document}