\documentclass[12pt,ngerman,parskip=half]{scrartcl}

\usepackage[utf8]{inputenc} % Eingabekodierung
\usepackage[T1]{fontenc} % Schriftkodierung 
\usepackage{booktabs} % schöne Tabellen
\usepackage{babel} % Silbentrennung und ``Eindeutschung''
\usepackage{graphicx} % Grafiken einbinden
\usepackage{csquotes} % Zitate
\usepackage{paralist} % Kompakte Aufzählungen
\usepackage{xcolor} % Farben
\usepackage{amsmath} % Erweiterung des Mathematik-Modus
\usepackage{blindtext} % Dummytext
\usepackage{microtype} %Mikrotypografie

% ohne Paket ==> Standard-Schrift = Computer Modern
%\usepackage{lmodern} % Moderne Version der Computer Modern

%\usepackage{fouriernc} % URW Schoolbook
%\usepackage{fourier} %Utopia Regular with Fourier
%\usepackage{concmath} % Concrete Math
%\usepackage[math]{iwona} % Iwona

% 
\usepackage{sansmathfonts}
%\usepackage[scaled=0.86]{beramono} % for code listings
\usepackage[scaled=0.95]{helvet}
\renewcommand{\rmdefault}{\sfdefault}

%Beispiel, wenn Font keinen Mathefont hat
%\usepackage{aurical} in Präambel,  \Fontauri im Dokument


\title{Schriftarten in pdf\LaTeX}
\author{Uwe Ziegenhagen}

\begin{document}
\maketitle


\blindtext

\begin{equation}
x_{1,2} = -\frac{p}{2} \pm \sqrt{  \left( \frac{p}{2} \right) ^2 - q  }
\end{equation}


\blindtext

\begin{equation}
\left\{
\begin{array}{l}
p = - \frac{5}{3} P^3 \left\{A \frac{\delta^2 \left(\frac{1}{\rho} \right)}{\delta \xi^2} + B  \frac{\delta^2 \left(\frac{1}{\rho} \right)}{\delta \eta^2} + C \frac{\delta^2 \left(\frac{1}{\rho} \right)}{\delta \zeta^2} \right\} + \text{konst.}, \\
u = A \xi - \frac{5}{3} P^2 A \frac{\xi}{\rho^3} - \frac{\delta D}{\delta \xi}, \\
v = B \eta - \frac{5}{3} P^3 B \frac{\eta}{\rho^3} - \frac{\delta D}{\delta \eta}, \\
w = C \zeta - \frac{5}{3} P^3 C \frac{\zeta}{\rho^3} - \frac{\delta D}{\delta \zeta},
\end{array}
\right.
\end{equation}


\blindtext

\blindtext


\end{document}