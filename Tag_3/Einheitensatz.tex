\documentclass[20pt,ngerman]{scrartcl}

\usepackage[utf8]{inputenc}
\usepackage[T1]{fontenc}
\usepackage{booktabs}
\usepackage{babel}
\usepackage{graphicx}
\usepackage{csquotes}
\usepackage{paralist}
\usepackage{xcolor}

\usepackage{siunitx}

\begin{document}

$O(n\log{}n)$ % regular O
 
$\mathcal{O}(n\log{}n)$ % Open at top left

%https://texblog.org/2014/06/24/big-o-and-related-notations-in-latex/


Hallo, ich bin ein \ang{90,528} \ang{90.528} Winkel.

Hallo, wir sind Zahlen: \num{3,1415927} \num{10,456} \num{1000000}.

\si{m^2}

\si{kg.m.s^{-1}}

\si{\metre \per  \second^2}

\si{\square\volt\cubic\lumen\per\farad}

% Zahlen mit Einheiten über \SI{} statt \si{}

\SI{9,81}{m.s^{-2}} \SI{9,81}{m\per \second^2}

\SI{3,1415927}{m^2}

% Listen von Zahlen und Einheiten

\numlist{1;2;3;4;5}

\numrange{1}{100}

\SIlist{1;2;3}{m^2}

\SIrange{1}{10}{m^3}

\num{1000000}

1\,000\,000

\vspace*{1cm}
\begin{tabular}{lrS}
1.23456 & 1000     & 10.34 \\
1.23456 & 10000   & 1150.3 \\ 
1.23456 & 100000 & 10.35345 \\
\end{tabular}



\end{document}