\documentclass[ngerman]{beamer}
% Option handout um Version ohne \pause etc. zu erzeugen

\usepackage[utf8]{inputenc}
\usepackage[T1]{fontenc}
\usepackage{babel}

\usepackage{booktabs}
\usepackage{graphicx}
\usepackage{csquotes}
\usepackage{xcolor}
\usepackage{listings}
 
\lstset{literate=%
    {Ö}{{\"O}}1
    {Ä}{{\"A}}1
    {Ü}{{\"U}}1
    {ß}{{\ss}}1
    {ü}{{\"u}}1
    {ä}{{\"a}}1
    {ö}{{\"o}}1
    {~}{{\textasciitilde}}1
}

\definecolor{colBack}{rgb}{1,1,0.9}
\definecolor{colKeys}{rgb}{0,0,1}
\definecolor{colIdentifier}{rgb}{0,0,0}
\definecolor{colComments}{rgb}{1,0,0}
\definecolor{colString}{rgb}{0,0.5,0}

\lstset{%
    float=hbp,%
    basicstyle=\ttfamily\footnotesize, %
    identifierstyle=\color{colIdentifier}, %
    keywordstyle=\color{colKeys}, %
    stringstyle=\color{colString}, %
    commentstyle=\color{colComments}, %
    columns=flexible, %
    tabsize=2, %
    frame=single, %
    extendedchars=true, %
    showspaces=false, %
    showstringspaces=false, %
    numbers=left, %
    numberstyle=\tiny, %
    breaklines=true, %
    backgroundcolor=\color{colBack}, %
    breakautoindent=true, %
    captionpos=b%
}



%\usetheme{PaloAlto}

\author{Uwe Ziegenhagen}
\title{Meine erste Präsentation}
\subtitle{Meine Abenteuer in Entenhausen}
\date{Köln, den \today}

\begin{document}

\begin{frame}

\maketitle

\end{frame}

\begin{frame}
\frametitle{Unser heutiges Programm}
\framesubtitle{Tag 1}

\tableofcontents

\end{frame}





\section{Grundlagen}

\frame{
\frametitle{Hallo Fernuni Hagen!}

\begin{itemize}
	\item Hallo
	\item ich 
	\item bin 
	\item eine 
	\item Bullet
	\item Liste
\end{itemize}

}


\frame{
\frametitle{Hallo Fernuni Hagen!}

\begin{enumerate}
	\item Hallo
	\item ich 
	\item bin 
	\item eine 
	\begin{enumerate}
	\item Hallo
	\item ich 
	\item bin 
	\item eine 
	\item Bullet
	\item Liste
\end{enumerate}
	\item Bullet
	\item Liste
\end{enumerate}

}


\begin{frame}
\frametitle{Bilder gehen auch}

\begin{figure}[htb]
	\centering
	\includegraphics[width=\textwidth]{miau.jpg}
	\caption{Das ist eine Katze}\label{fig:mieze2}
\end{figure}

\end{frame}


\begin{frame}
\frametitle{Zweispaltig}

\begin{columns}

\begin{column}{0.33\textwidth}
	\begin{itemize}
		\item Ich bin 
		\item eine 
		\item Bullet-Liste
	\end{itemize}
\end{column}

\begin{column}{0.66\textwidth}
	\begin{figure}[htb]
		\centering
		\includegraphics[width=0.9\textwidth]{miau.jpg}
		\caption{Das ist eine Katze}\label{fig:mieze2}
	\end{figure}
\end{column}

\end{columns}
\end{frame}

\begin{frame}
\frametitle{Mathematik}

\[a^2 + b^2 = c^2 \]

\begin{equation}
a^2 + b^2 = c^2
\end{equation}


\end{frame}

\begin{frame}

\begin{itemize}
	\item Hallo \pause
	\item ich \pause
	\item bin  \pause
	\item eine \pause 
	\item Bullet
	\item Liste
\end{itemize}

\end{frame}

\section{Aufdeckungen}

\begin{frame}

\begin{itemize}
	\item<1-> Hallo 
	\item<2> ich 
	\item<-2> bin  
	\item<3-> eine 
	\item<1> Bullet
	\item<2> Liste
\end{itemize}

\end{frame}


\begin{frame}

\begin{itemize}
	\item<1-> Hallo 
	\item<2-> ich 
	\item<3-> bin  
	\item<4-> eine 
	\item<5-> Bullet
	\item<6-> Liste
\end{itemize}

\end{frame}

\begin{frame}[fragile]

\begin{lstlisting}[language={Python}]
# delete copied files
unlink('./dtk-bibliography/dtk-authoryear.bbx')
unlink('./dtk-bibliography/dtk-authoryear.dbx')
unlink('./dtk-bibliography/dtk-bibliography.pdf')
unlink('./dtk-bibliography/dtk-bibliography.tex')
\end{lstlisting}

\end{frame}



\end{document}