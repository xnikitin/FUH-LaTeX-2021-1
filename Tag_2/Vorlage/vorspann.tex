\usepackage[utf8]{inputenc} % Eingabekodierung ist utf8
\usepackage[T1]{fontenc} % Westeuropäisch
\usepackage{babel} % Eindeutschung von Begriffen wie Inhaltsverzeichnis etc
\usepackage{csquotes} % Text in Gänsefüßchen setzen
\usepackage{booktabs} % schöne Tabellen
\usepackage{graphicx} % Bilder einfügen
\usepackage{xcolor} % für Farben
\usepackage{paralist} % für eng gesetzte Aufzählungen

\usepackage{blindtext}  % für Dummy Test

\usepackage{microtype}

\usepackage{subcaption} % Mehrere Bilder nebeneinander mit unterschiedlichen Captions


\newcommand{\person}[1]{\textsc{#1}}






% Für Hyperlinks, Paket immer am Ende laden
\usepackage{hyperref}
\hypersetup{
    bookmarks=true,                     % show bookmarks bar
    unicode=false,                      % non - Latin characters in Acrobat’s bookmarks
    pdftoolbar=true,                        % show Acrobat’s toolbar
    pdfmenubar=true,                        % show Acrobat’s menu
    pdffitwindow=false,                 % window fit to page when opened
    pdfstartview={FitH},                    % fits the width of the page to the window
    pdftitle={My title},                        % title
    pdfauthor={Author},                 % author
    pdfsubject={Subject},                   % subject of the document
    pdfcreator={Creator},                   % creator of the document
    pdfproducer={Producer},             % producer of the document
    pdfkeywords={keyword1, key2, key3},   % list of keywords
    pdfnewwindow=true,                  % links in new window
    colorlinks=true,                        % false: boxed links; true: colored links
    linkcolor=red,                          % color of internal links
    filecolor=cyan,                     % color of file links
    citecolor=green,                     % color of file links
    urlcolor=magenta                        % color of external links
}
