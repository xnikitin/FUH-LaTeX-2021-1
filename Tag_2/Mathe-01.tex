\documentclass[12pt,ngerman,parskip=full]{scrreprt}

\usepackage[utf8]{inputenc}
\usepackage[T1]{fontenc}
\usepackage{booktabs}
\usepackage{babel}
\usepackage{graphicx}
\usepackage{csquotes}
\usepackage{paralist}
\usepackage{xcolor}

% Für den Mathesatz
\usepackage{amsmath}
\usepackage{derivative}
\usepackage{esvect} % für \vv{} Vektorbefehl
\DeclareMathOperator{\xyz}{xyz}

\begin{document}

\chapter{Einleitung}

Hallo, ich bin eine Formel $a^2+b^2=c^2$  im fließenden Text. 

\[ a^2 + b^2 = c^2 \]

$$ a^2 + b^2 = c^2 $$ sollte man nicht mehr benutzen, das dies TeX-Notation ist und nicht offiziell in \LaTeX\ unterstützt wird. 

Symbole findet man am einfachsten, wenn man über die Kommandozeile \enquote{texdoc symbols} aufruft.

\begin{equation}\label{eq:pythagoras}
a^2 + b^2 = c^2 \Rightarrow c = \sqrt{a^2+b^2} \rightarrow  = \sqrt[2]{a^2+b^2}
\end{equation}

Siehe Gleichung  \ref{eq:pythagoras} auf Seite \pageref{eq:pythagoras}.

\[ a^2 + b_3^2  \not=  a^{2^2} + b_{3^2}  \]

\[  \sum_{i=1}^{100} = 5050  \rightarrow \sum_{i=\infty}^{\infty} = \infty \]

\[  -\frac{p}{2} \pm \sqrt{  \left(\frac{p}{4}\right)^2  - q  }   \]

\[  \overbrace{a^2 + b^2} = \underbrace{c^2}   \]

\[ \prod_{x}^{y} = x \times y \cdot z  \mod t \]


\[  \frac{\delta x}{\delta y} = c^2   \]

\[ \odv*[3]{y}{x} +  \pdv*{f}{x,y} \]

\[  \int_{x=1}^{\infty} \sin x \cos y \tan z = 4 \xyz 584 \]

\begin{eqnarray}
y &=& (a+b)^2 \\
y &=& a^2 + 2ab + b^2  
\end{eqnarray}

\[
\begin{array}{rcl}
y &=& (a+b)^2 \\
y + c+ e+ r &=& a^2 + 2ab + b^2  
\end{array}
\]

\[\bordermatrix{%
  & 1 & 2 & 3 \cr
1& 1 & 2 & 3 \cr
2& 1 & 2 & 3 \cr
3 & 1 & 2 & 3 \cr
}\]

% Standard in LaTeX, amsmath hat weitere Typen
\[\left(
\begin{array}{ccc} 
1 & 0 & 0 \\ 
0 & 1 & 0 \\ 
0 & 0 & 1 \\ 
\end{array}\right)
\]

\[\left[
\begin{array}{ccc} 
1 & 0 & 0 \\ 
0 & 1 & 0 \\ 
0 & 0 & 1 \\ 
\end{array}\right]
\]

%amsmath - Matrix ohne Klammern
\[
\begin{matrix} 
1 & 0 & 0 \\ 
0 & 1 & 0 \\ 
0 & 0 & 1 \\ 
\end{matrix}
\]

%amsmath - Matrix mit runden Klammern
\[
\begin{pmatrix} 
1 & 0 & 0 \\ 
0 & 1 & 0 \\ 
0 & 0 & 1 \\ 
\end{pmatrix}
\]

%amsmath - Matrix mit eckigen Klammern
\[
\begin{bmatrix} 
1 & 0 & 0 \\ 
0 & 1 & 0 \\ 
0 & 0 & 1 \\ 
\end{bmatrix}
\]

%amsmath - Matrix mit geschweiften Klammern
\[
\begin{Bmatrix} 
1 & 0 & 0 \\ 
0 & 1 & 0 \\ 
0 & 0 & 1 \\ 
\end{Bmatrix}
\]


%amsmath - Matrix mit geschweiften Klammern
\[ X = \det 
\begin{pmatrix} 
1 & 0 \\ 
0 & 1 \\ 
\end{pmatrix}
\]

%amsmath - Matrix mit einfachen senkrechten Strichen
\[ z = \det 
\begin{vmatrix} 
1 & 0 & 0 \\ 
0 & 1 & 0 \\ 
0 & 0 & 1 \\ 
\end{vmatrix}
\]

%amsmath - Matrix mit doppelten senkrechten Strichen
\[ z = \alpha \beta \gamma \pi 
\begin{Vmatrix} 
1 & 0 & 0 \\ 
0 & 1 & 0 \\ 
0 & 0 & 1 \\ 
\end{Vmatrix}
\]


\[ \vv{a}  \cdot \vv{abc} \]

\[  a \cap b  \cup c \rightarrow {x | x \text{ ist positiv}} \]

\[ A \not= \overline{A} \]

\[ ABC \not= \overline{ABC}  \Omega \]

\clearpage

Abstände im Mathematiksatz

\[ a b c \rightarrow abc \]

\[ a b c \rightarrow a\,b\,c \]

\[ a b c \rightarrow a\;b\;c \]


\[ a b c \rightarrow a\quad b\quad c \]

\[ a b c \rightarrow a\qquad b\qquad c \]





\end{document}